\documentclass{article}
\usepackage[utf8]{inputenc}
\usepackage[margin=0.8in]{geometry}
\usepackage[english]{babel}
\usepackage{graphicx}
\usepackage{listings} % linebreaks in lstlisting
\usepackage{minted} % python code


\lstset{
basicstyle=\small\ttfamily,
columns=flexible,
breaklines=true
}
\begin{document}

\title{Group 3 Project Part 2}
\author{
    Kayla Wankowski (kmw32@students.uwf.edu)\\
    Muddasar Ali (ma229@students.uwf.edu)\\
    Raffaele Galliera (rg101@students.uwf.edu)\\
    Tobias Jacob (tj75@students.uwf.edu)
}

\maketitle

\section{Data cubing}

\begin{lstlisting}
    Game:  TEAM_ID_HOME, TEAM_NAME_HOME, TEAM_ID_AWAY, TEAM_NAME_AWAY, GAME_DATE, SEASON, HOME_TEAM_WINS, HOME_TEAM_LOSSES, PTS_HOME, PTS_AWAY 
    Player_Attributes:  ID, FIRST_NAME, LAST_NAME, BIRTHDATE, HEIGHT, WEIGHT, PTS, TEAM_ID, TEAM_NAME, FROM_YEAR, TO_YEAR 
    Team_Attributes:  ID, YEARFOUNDED, CITY, NICKNAME 
\end{lstlisting}

\subsection{Cube of games}
With the given table, we construct the \texttt{cubeGames}.
\begin{lstlisting}
Dimensions:
- TEAM_NAME_HOME
- TEAM_NAME_AWAY
- GAME_DATE < SEASON
Values:
- HOME_TEAM_WINS
- HOME_TEAM_LOSSES
- PTS_HOME
- PTS_AWAY
- FGM_HOME
- FGA_HOME
- FG3M_HOME
- FG3A_HOME
- FTM_HOME
- FTA_HOME
- FGM_AWAY
- FGA_AWAY
- FG3M_AWAY
- FG3A_AWAY
- FTM_AWAY
- FTA_AWAY
- YEARFOUNDEDHOME_TEAM
- CITYHOME_TEAM
- NICKNAMEHOME_TEAM
- YEARFOUNDEDAWAY_TEAM
- CITYAWAY_TEAM
- NICKNAMEAWAY_TEAM
\end{lstlisting}
\texttt{GAME\_DATE} and \texttt{SEASON} are a concept hierarchy.
\texttt{HOME} means the team at whose stadion the game happens.
\texttt{AWAY} is the team that visits.
\texttt{WINS} and \texttt{LOSSES} specifiy the rounds won or lost.
\texttt{PTS} are the points scored.
\texttt{FGM} is the number of field goals (2 and 3pts) made.
\texttt{FG3M} is the number of 3pt field goals attempted.
\texttt{FA} is the number of field goals made.
\texttt{FG3A} is the number of 3pt field goals attempted.
\texttt{FTA} is the number free throws attempted.
\texttt{FTM} is the number free throws made. Free throws are worth one point.
The other columns are self-explainatory.

\subsection{Cube of players}
This is the construction of \texttt{cubePlayers}.
\begin{lstlisting}
Dimensions
- FULL_NAME < TEAM_NAME
- POSITION
- AGE
- FROM_YEAR_AGE
- TO_YEAR_AGE

Values
- FIRST_NAME
- LAST_NAME
- HEIGHT
- WEIGHT
- PTS
- FROM_YEAR
- TO_YEAR
- value
- YEARFOUNDED
- CITY
- NICKNAME
- BIRTHDATE
- ACTIVE_YEARS
\end{lstlisting}
\texttt{FULL\_NAME} and \texttt{TEAM\_NAME} form a concept hierarchy.
\texttt{FROM\_YEAR} is when the player started playing in the NBA with the age \texttt{FROM\_YEAR\_AGE}.
\texttt{TO\_YEAR} specifies when the player left the NBA.
\texttt{value} is the value of the player.
\texttt{ACTIVE\_YEARS} is the time the player spend at the NBA.


\subsection{Cube of Draft}
This is the construction of \texttt{cubeDraft}.
\begin{lstlisting}
Dimensions
- FULL_NAME < TEAM_NAME
- POSITION
- AGE
- FROM_YEAR_AGE
- TO_YEAR_AGE

Values
- FIRST_NAME
- LAST_NAME
- HEIGHT
- WEIGHT
- PTS
- FROM_YEAR
- TO_YEAR
- value
- YEARFOUNDED
- CITY
- NICKNAME
- BIRTHDATE
- ACTIVE_YEARS
\end{lstlisting}
\texttt{FULL\_NAME} and \texttt{TEAM\_NAME} form a concept hierarchy.
\texttt{FROM\_YEAR} is when the player started playing in the NBA with the age \texttt{FROM\_YEAR\_AGE}.
\texttt{TO\_YEAR} specifies when the player left the NBA.
\texttt{value} is the value of the player.
\texttt{ACTIVE\_YEARS} is the time the player spend at the NBA.

\subsection{Biometrics cube}
For later project submissions, categorial data attributes are needed.
We construct a cube with biometric data of the players.
To discretize scalar values like \texttt{HEIGHT} or \texttt{WEIGHT}, we utilize 1 dimensional k-Means clustering, which is already implemented.
\begin{lstlisting}
Dimensions:
- FULL_NAME < TEAM_NAME
- POSITION: [Guard, Forward-Guard, Forward, Forward-Center, Center]
- ACTIVE_YEARS: [Short, Middle, Long]

Values:
- HEIGHT: [Short, Middle, Tall]
- WEIGHT: [Light, Middle, Heavy]
- PTS: [LessScores, MiddleScores, MoreScores, ManyScores]
\end{lstlisting}
This cube allows us to calculate t- and d-weights.

\section{Figures}
We added some interesting figures

\begin{figure}[h]
    \centering
    \includegraphics[width=8cm]{figures/ACTIVE_YEARS_over_PTS.eps}
    \caption{Scored points over active years based on player position.}
\end{figure}

\begin{figure}[h]
    \centering
    \includegraphics[width=8cm]{figures/ACTIVE_YEARS_over_PTS2.eps}
    \caption{Regression on points per active year.}
\end{figure}

\begin{figure}[h]
    \centering
    \includegraphics[width=8cm]{figures/HEIGHT_over_PTS2.eps}
    \caption{Taller player does not necessarily mean higher scores.}
\end{figure}

\begin{figure}[h]
    \centering
    \includegraphics[width=8cm]{figures/NBA_GROWTH.eps}
    \caption{How the nba grew over time.}
\end{figure}

\begin{figure}[h]
    \centering
    \includegraphics[width=8cm]{figures/START_VS_END_AGE.eps}
    \caption{At which age players join and leave the NBA.}
\end{figure}

\begin{figure}[h]
    \centering
    \includegraphics[width=8cm]{figures/DraftRound_over_PTS2.eps}
    \caption{How does draft round affect points scored.}
\end{figure}

\clearpage

\section{Result from clustering}

We discretized values.
To get a better understanding, we visualized each dimension that was discretized.
We show the distribution of the values, and the cluster centers at the bottom of the graph.

\begin{figure}[h]
    \centering
    \includegraphics[width=8cm]{figures/disc/ACTIVE_YEARS.eps}
    \caption{Distribution of active years and cluster centers.}
\end{figure}
\begin{figure}[h]
    \centering
    \includegraphics[width=8cm]{figures/disc/HEIGHT.eps}
    \caption{Distribution of player height and cluster centers.}
\end{figure}
\begin{figure}[h]
    \centering
    \includegraphics[width=8cm]{figures/disc/PTS.eps}
    \caption{Distribution of player points and cluster centers.}
\end{figure}
\begin{figure}[h]
    \centering
    \includegraphics[width=8cm]{figures/disc/WEIGHT.eps}
    \caption{Distribution of player weight and cluster centers.}
\end{figure}

\end{document}
